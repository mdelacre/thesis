\clearpage



\begin{appendix}
\hypertarget{annexes}{%
\section{Annexes}\label{annexes}}

\hypertarget{annexe-a-erratum-des-articles}{%
\subsection{Annexe A: erratum des
articles}\label{annexe-a-erratum-des-articles}}

\hypertarget{why-psychologists-should-by-default-use-welchs-t-test-instead-of-students-t-test-chapitre-2}{%
\subsubsection{\texorpdfstring{Why psychologists Should by Default Use
Welch's \emph{t}-test Instead of Student's \emph{t}-test (Chapitre
2)}{Why psychologists Should by Default Use Welch's t-test Instead of Student's t-test (Chapitre 2)}}\label{why-psychologists-should-by-default-use-welchs-t-test-instead-of-students-t-test-chapitre-2}}

\hypertarget{erreurs-conceptuelles}{%
\paragraph{Erreurs conceptuelles}\label{erreurs-conceptuelles}}

Nous spécifions Ã~ plusieurs reprises que le test \(t\) de Yuen
contrôle moins bien le taux d'erreur de type I que le test \(t\) de
Welch:\\
- p.14: \emph{``Yuen's \(t\)-test is not a good unconditional
alternative because we observe an unacceptable departure from the
nominal alpha risk of 5 percent for several shapes of distributions
{[}\ldots{]} particularly when we are studying asymmetric distributions
of unequal shapes''};\\
- p.15: \emph{``As it is explained in the additional file, Yuen's
\(t\)-test is not a better test than Welch's \(t\)-test, since it often
suffers high departure from the alpha risk of 5 percent''}.

Ceci n'est pas exact d'un point de vue purement statistique. A travers
le test de Yuen, on ne compare plus les moyennes de chaque groupe, mais
les moyennes \emph{trimmées} (soit les moyennes calculées sur les
données après avoir écarté les 20\% des scores les plus bas ainsi
que les 20\% des scores les plus élevés). Or, Ã~ travers nos
simulations, les scénarios créés en vue de tester le taux d'erreur de
type I (risque alpha) étaient systématiquement des scénarios dans
lesquels les moyennes de chaque groupe étaient identiques. Lorsque les
échantillons sont extraits de population parfaitement symmétriques, il
y aura vraisemblablement peu de différence entre les estimations des
moyennes et des moyennes trimmées (puisqu'au niveau de la population,
les moyennes et moyennes trimmées seront identiques). Quand les
échantillons sont extraits de population asymétriques, par contre, on
observera plus vraisemblablement des différences entre les estimations
des moyennes et des moyennes trimmées. Au niveau de la population, les
moyennes trimmées seront plus proches du mode des distributions que les
moyennes et représenteront donc mieux les distributions.

Notons malgré tout que d'un point de vue méthodologique, nous avons
déjÃ~ relevé que la plupart du temps, les chercheurs définissent
l'absence de différence entre les moyennes comme hypothèse nulle et
dans ce contexte, nos simulations démontre que le test de Yuen n'est
pas approprié. En conclusion, le test de Yuen ne devrait être utilisé
que par des chercheurs ayant pleinement conscience du fait que les test
\(t\) de Student et de Welch ne testent pas la même hypothèse que le
test \(t\) de Yuen.

\hypertarget{commentaires-divers}{%
\paragraph{Commentaires divers}\label{commentaires-divers}}

\begin{itemize}
\item
  p.9: nous décrivons 3 arguments en défaveur de l'usage du test de
  Levene. En troisième argument, nous mentionnons le manque de
  puissance du test de Levenne. Nous ne mentionnons cependant pas le
  fait qu'utiliser le test \(t\) de Student lorsque le test de Levene
  est non significatif revient Ã~ confondre le non rejet de l'hypothèse
  d'égalité des variances avec l'acceptation de l'hypothèse
  d'égalité des variances. Au sein du chapitre 5 sur les tests
  d'équivalence, il est démontré par simulation que même lorsqu'on
  s'assure d'avoir une puissance suffisante pour détecter une
  différence attendue, la stratégie qui consiste Ã~ interpréter le
  non rejet de l'hypothèse nulle comme un soutien en faveur de
  l'hypothèse nulle n'est pas appropriée.
\item
  p.12: nous mentionnons ceci : \emph{``When both variances and sample
  sizes are the same in each independent group, the \(t\)-values,
  degrees of freedom, and the \(p\)-values in Student's \(t\)-test and
  Welch's \(t\)-test are the same (see Table 1)\emph{. Par "variances"
  il faut comprendre "}sample* variances'' ou ``variances
  \emph{estimates}''. Nous ne sommes donc }pas* en train de dire que les
  deux statistiques, ainsi que les degrés de liberté et \(p\)-valeurs
  qui leur sont associées seront identiques lorsque la condition
  d'homogénéité des variances sera respectée au niveau de la
  population, mais bien lorsque les estimations de chaque variance de
  population seront identiques.
\end{itemize}

\hypertarget{mise-en-forme-et-notations}{%
\paragraph{Mise en forme et
Notations}\label{mise-en-forme-et-notations}}

Les lettres utilisées pour décrire les statistiques (test-\(t\) ou
test-\(F\)) doivent toujours être inscrites en \emph{italique}. Or,
cela a été omis Ã~ plusieurs reprises dans l'article. Par exemple, il
aurait fallu écrire:\\
- p.9: ``\ldots{} as the Mann-Whitney \(U\)-test\ldots{}'' au lieu de
``\ldots{} as the Mann-Whitney U-test\ldots{}'';\\
- p.9: ``\(F\)-ratio test'' au lieu de ``F-ratio test''.\\
\newpage Certaines notations mathématiques auraient également dû
être indiquées en italique. Par exemple, Ã~ la p.9, il aurait fallu
écrire:\\
- ``\(x_{ij}\)'' au lieu de ``\(\mathrm{x_{ij}}\)'';\\
- \textbar{}\(x_{ij}-\hat{\theta_j}\)\textbar{} au lieu de
\textbar{}\(\mathrm{x_{ij}-\hat{\theta_j}}\)\textbar.

Par ailleurs, il est très important d'être consistant dans le choix
des notations mathématiques, afin d'éviter d'embrouiller le lecteur.
Or, nous n'avons pas toujours respecté cela. Par exemple, nous avons
utilisé plusieurs notations différentes pour décrire l'écart-type/la
variance. Par exemple:\\
- p.9: nous utilisons respectivement SD1 et SD2 pour décrire
l'écart-type de chaque groupe;\\
- p.11 (équation 1): nous utilisons respectivement \(S^2_1\) et
\(S^2_2\) pour décrire la variance de chaque groupe;\\
- p.12 (équation 4): nous utilisons respectivement \(s^2_1\) et
\(s^2_2\) (lettres minuscules) pour décrire la variance de chaque
groupe.

C'est d'autant plus problématique qu'il y a parfois même des
inconsistances entre les notations utilisées dans les formules et
celles utilisées dans les légendes des formules. Par exemple, nous
spécifions p.11 que dans l'équation 1, \(s^2_1\) et \(s^2_2\) (lettres
minuscules) représentent les estimations de variance de chaque groupe
indépendant, alors qu'en réalité, les estimations des variances sont
représentées par \(S^2_1\) et \(S^2_2\) (lettres majuscules) dans
l'équation 1.

Notons pour finir une faute de frappe Ã~ la p.13: nous indiquons ``see v
2a'' au lieu de ``see Figure 2a''.

\hypertarget{taking-parametric-assumptions-very-seriously-arguments-for-the-use-of-welchuxe2s-f-test-instead-of-the-classical-f-test-in-one-way-anova-chapitre-3}{%
\subsubsection{\texorpdfstring{Taking parametric assumptions very
seriously: Arguments for the Use of Welch’s \emph{F}-test instead of
the Classical \emph{F}-test in One-Way ANOVA (Chapitre
3)}{Taking parametric assumptions very seriously: Arguments for the Use of Welch’s F-test instead of the Classical F-test in One-Way ANOVA (Chapitre 3)}}\label{taking-parametric-assumptions-very-seriously-arguments-for-the-use-of-welchuxe2s-f-test-instead-of-the-classical-f-test-in-one-way-anova-chapitre-3}}

\hypertarget{erreurs-conceptuelles-1}{%
\paragraph{Erreurs conceptuelles}\label{erreurs-conceptuelles-1}}

\hypertarget{mise-en-forme-et-notations-1}{%
\paragraph{Mise en forme et
Notations}\label{mise-en-forme-et-notations-1}}

Dû Ã~ un manque de connaissance de Latex lors de mes premières
tentatives d'écritures d'articles via Rmarkdown, certaines majuscules
sont manquantes dans les références bibliographiques. S'assurer qu'une
lettre apparaissent en majuscule, via latex, implique de l'entourer des
symboles \{\}, ce qui n'a pas été fait. Par exemple, dans le titre de
l'article de Tiku(1971), il aurait fallu indiquer ``Power function of
the \{F\}-test..'' au lieu de ``Power function of the F-test\ldots{}''.
Cela ne serait pas arrivé, si j'avais utilisé un outil comme Zotero,
afin d'exporter directement un fichier au format Bibtex (puisque via ces
outils, ce genre de détail est automatiquement inclu), mais je n'ai
découvert cette possibilité que très récemment.

\hypertarget{effect-sizes}{%
\subsubsection{Effect sizes}\label{effect-sizes}}

\hypertarget{equivalence-tests}{%
\subsubsection{Equivalence tests}\label{equivalence-tests}}

\newpage

\hypertarget{annexe-b}{%
\subsection{Annexe B}\label{annexe-b}}

Insert code (if any) used during your dissertation work here.
\end{appendix}
