\clearpage



\begin{appendix}
\hypertarget{erratum}{%
\section{Erratum}\label{erratum}}

\hypertarget{why-psychologists-should-by-default-use-welchs-t-test-instead-of-students-t-test-chapitre-2}{%
\subsection{\texorpdfstring{``Why psychologists Should by Default Use
Welch's \emph{t}-test Instead of Student's \emph{t}-test'' (Chapitre
2)}{``Why psychologists Should by Default Use Welch's t-test Instead of Student's t-test'' (Chapitre 2)}}\label{why-psychologists-should-by-default-use-welchs-t-test-instead-of-students-t-test-chapitre-2}}

\hypertarget{erreurs-conceptuelles}{%
\subsubsection{Erreurs conceptuelles}\label{erreurs-conceptuelles}}

\begin{enumerate}
\def\labelenumi{\arabic{enumi})}
\tightlist
\item
  \textbf{p.10}: \emph{``the \(F\)-ratio statistic is obtained by
  computing SD2/SD1 (standard deviation ratio, SDR)''}
\end{enumerate}

\begin{itemize}
\tightlist
\item
  D'abord, le ratio entre les 2 écart-types d'échantillons ne
  correspond pas au SDR, mais Ã~ l'\textbf{estimation} du SDR (SDR =
  \(\frac{\sigma_2}{\sigma_1}\)).\\
\item
  Ensuite, ce n'est pas nécessairement le deuxième écart-type
  d'échantillon qui se trouve au numérateur, mais le plus grand des
  deux échantillons (si bien que le \(F\)-ratio est toujours supérieur
  Ã~ 1; Hayes \& Cai, 2007).
\end{itemize}

\begin{enumerate}
\def\labelenumi{\arabic{enumi})}
\setcounter{enumi}{1}
\item
  \color{black}\textbf{p.10}: ``When SDR = 1, the equal variances
  assumption is true, when SDR \textgreater{} 1 the standard deviation
  of the second \sout{sample} \color{blue} population \color{black} is
  bigger than the standard deviation of the first \sout{sample}
  \color{blue} population \color{black}, and when SDR \textless{} 1 the
  standard deviation of the second \sout{sample} \color{blue}population
  \color{black} is smaller than the standard deviation of the first
  \sout{sample} \color{blue} population''. \color{black}
\item
  \textbf{p.14} (partie ``simulations'') : \emph{``As long as the
  variances are equal between \sout{groups} \color{blue} populations
  \color{black} or sample sizes are equal, the distribution of
  Student’s \(p\)-values is uniform\ldots{}''}.
\end{enumerate}

Note: Ã~ divers autres endroits, j'ai remplacé le mot ``groupe'' par le
mot ``population'' pour éviter toute ambiguïté (car j'utilisais
parfois le mot ``groupe'' pour décrire les échantillons, parfois pour
décrire les populations).

\begin{enumerate}
\def\labelenumi{\arabic{enumi})}
\setcounter{enumi}{3}
\item
  \textbf{p.14} (partie ``simulations'') :\emph{``Differences are small,
  except in three scenarios (See table A5.2, A5.5 , and A5.6 in the
  additionnal file)}. En réalité, la plus grande différence entre les
  tests se trouve en A5.7 (quand les deux distributions présentent une
  asymétrie positive) et vaut 4.29\%. Dans tous les autres cas, elle
  est inférieure. Le constat erroné d'après laquelle la différence
  dans les tables A5.2, A5.5 et A5.6 était plus marquée provenait
  d'une erreur dans mes scripts, qui a été corrigée (également sur
  le Supplemental Material attaché en ligne Ã~ l'article du chapitre
  2).
\item
  \textbf{Test de Yuen}: nous sous-entendons Ã~ plusieurs reprises que
  le test de Yuen a un taux inacceptable d'erreur de type I lorsque les
  distributions de population sont asymétriques. Il s'agit d'une erreur
  conceptuelle qui a été expliquée dans les limites de la thèse.
\end{enumerate}

\hypertarget{ambiguuxe3tuxe3s-possibles}{%
\subsubsection{Ambiguïtés
possibles}\label{ambiguuxe3tuxe3s-possibles}}

\begin{enumerate}
\def\labelenumi{\arabic{enumi})}
\item
  \textbf{p.13} : \emph{``When both variances and sample sizes are the
  same in each independent group, the \(t\)-values, degrees of freedom,
  and the \(p\)-values in Student's \(t\)-test and Welch's \(t\)-test
  are the same (see Table 1).''} Les statistiques, degrés de liberté
  et \(p\)-valeurs seront identiques Ã~ condition que les \color{blue}
  estimations de variances soient identiques \color{black} et soient
  obtenues sur base \color{blue} d'échantillons de tailles
  égales\color{black}. Or, la phrase peut donner l'impression que ces
  paramètres seront identiques Ã~ condition que la condition
  d'homoscédasticité soit respectée (ce qui n'est pas vrai). Cette
  information fournie dans la table est peu pertinente en soi, dans la
  mesure où très fréquemment, on obtiendra des estimations de
  variance différentes pour chaque groupe lorsque la condition
  d'homoscédasticité est respectée (et qu'Ã~ l'inverse, il est
  possible, bien que peu probable, d'obtenir des estimations de variance
  identiques même en cas d'hétéroscédasticité).
\item
  \textbf{p.16} (Ã~ propos du test de Levene) : \emph{``Because the
  statistical power for this test is often low, researchers will
  inappropriately choose Student's \(t\)-test instead of more robust
  alternatives.''} Cette phrase peut amener Ã~ comprendre que si le test
  de Levene était toujours très puissant, il serait approprié de
  l'utiliser en vue de choisir entre les tests \(t\) de Student et \(t\)
  de Welch. Pourtant, privilégier le test \(t\) de Student lorsque l'on
  ne peut rejeter l'hypothèse d'égalité des variances (autrement dit,
  lorsque les résultats du test de Levene sont non significatifs)
  reviendrait Ã~ confondre le non-rejet de l'hypothèse d'égalité des
  variances avec l'acceptation de l'hypothèse d'égalité des variances
  (cf.~chapitre 5).
\end{enumerate}

\hypertarget{mise-en-forme-et-notations}{%
\subsubsection{Mise en forme et
Notations}\label{mise-en-forme-et-notations}}

\begin{enumerate}
\def\labelenumi{\arabic{enumi})}
\tightlist
\item
  Les lettres utilisées pour décrire les statistiques (\(t\) ou \(F\),
  par exemple) doivent toujours être inscrites en \emph{italique}. Il
  en est de même pour les notations mathématiques. Or, cela a été
  omis Ã~ plusieurs reprises dans l'article. Par exemple, il aurait
  fallu écrire \ldots:
\end{enumerate}

\begin{itemize}
\tightlist
\item
  \textbf{p.10} : ``\ldots{} as the Mann-Whitney
  \color{blue}\(U\)\color{black}-test\ldots{}'';\\
\item
  \textbf{p.10} : ``\ldots the \color{blue}\(F\)\color{black}-ratio
  statistic\ldots{}'';\\
\item
  \textbf{p.10} : ``\ldots{} \(x_{ij}\)'' au lieu de
  ``\(\mathrm{x_{ij}}\)'';\\
\item
  \textbf{p.10} :
  ``\ldots\textbar{}\(x_{ij}-\hat{\theta_j}\)\textbar{}'' au lieu de
  ``\textbar{}\(\mathrm{x_{ij}-\hat{\theta_j}}\)\textbar{}''.
\end{itemize}

\begin{enumerate}
\def\labelenumi{\arabic{enumi})}
\setcounter{enumi}{1}
\tightlist
\item
  Il est très important d'être consistant dans le choix des notations
  mathématiques, pour éviter toute confusion pour le lecteur. Or, nous
  ne l'avons pas toujours été. Par exemple, nous avons utilisé
  plusieurs notations différentes pour décrire l'écart-type et la
  variance :
\end{enumerate}

\begin{itemize}
\tightlist
\item
  \textbf{p.10} : nous utilisons respectivement SD1 et SD2 pour décrire
  l'écart-type de chaque groupe;\\
\item
  \textbf{p.12 }(équation 1) : nous utilisons respectivement \(S^2_1\)
  et \(S^2_2\) pour décrire la variance de chaque groupe, alors que
  nous utilisons \(s^2_1\) et \(s^2_2\) (lettres minuscules) dans la
  légende de cette formule;\\
\item
  \textbf{p.13} (équations 3 et 4) : nous utilisons respectivement
  \(s^2_1\) et \(s^2_2\) (lettres minuscules) pour décrire la variance
  de chaque groupe.
\end{itemize}

\begin{enumerate}
\def\labelenumi{\arabic{enumi})}
\setcounter{enumi}{2}
\tightlist
\item
  On parle normalement d'erreur de type I et II. Or, dans tout l'article
  du chapitre 2, j'ai parlé des erreurs de type 1 et 2. Par exemple,
  p.13: \emph{``An increase in the Type 1 error rate leads to an
  inflation of\ldots{} while an increase in the Type 2 error
  rate\ldots{}''}
\end{enumerate}

\hypertarget{fautes-de-frappe}{%
\subsubsection{Fautes de frappe}\label{fautes-de-frappe}}

\begin{itemize}
\tightlist
\item
  \textbf{p.14} : ``see \sout{v} \color{blue}Figure \color{black} 2a''.
\item
  \textbf{p.15} : ``\(p\)-values from Welch's \sout{\(t\)-test} and
  Student's \(t\)-tests, shown separately\ldots{}''
\item
  \textbf{p.16} : Note 4 : ``other variants have been proposed such as
  the \color{blue}20\% \color{black} trimmed mean''
\end{itemize}

\hypertarget{bibliographie}{%
\subsubsection{Bibliographie}\label{bibliographie}}

Référence manquante: Bradley (1978).

\newpage

\hypertarget{annexe-b-erratum-de-larticle-taking-parametric-assumptions-very-seriously-arguments-for-the-use-of-welchuxe2s-w-test-instead-of-the-classical-f-test-in-one-way-anova-chapitre-3}{%
\subsection{\texorpdfstring{Annexe B: erratum de l'article ``Taking
parametric assumptions very seriously : Arguments for the Use of
Welch’s \emph{W}-test instead of the Classical \emph{F}-test in
One-Way ANOVA'' (Chapitre
3)}{Annexe B: erratum de l'article ``Taking parametric assumptions very seriously : Arguments for the Use of Welch’s W-test instead of the Classical F-test in One-Way ANOVA'' (Chapitre 3)}}\label{annexe-b-erratum-de-larticle-taking-parametric-assumptions-very-seriously-arguments-for-the-use-of-welchuxe2s-w-test-instead-of-the-classical-f-test-in-one-way-anova-chapitre-3}}

\hypertarget{erreur-conceptuelle}{%
\subsubsection{Erreur conceptuelle}\label{erreur-conceptuelle}}

\begin{itemize}
\tightlist
\item
  \textbf{p.22} : ``\sout{Formula (7)} \color{blue}Equation 7
  \color{black} provides the computation of the \sout{W-test, or
  Welch’s F-test} \color{blue}Welch’s statistic (\(W\)).
  \color{black} In the numerator of the \sout{W-test} \color{blue}\(W\)
  statistic,\color{black} the squared deviation between group
  mean\ldots{}''
\item
  \textbf{p.22} : ``\ldots{} negative pairings (the group with the
  \sout{smallest} \color{blue}largest \color{black} sample size is
  extracted from the population with the smallest \(SD\));
\item
  \textbf{p.25} : ``\ldots{} which is either more liberal or more
  conservative, depending on the \(SDs\) and \sout{\(SD\)}
  \color{blue}sample sizes \color{black} pairing'';
\item
  \textbf{p.27} : ``Moreover, there is one constant observation in our
  simulations : whatever the configuration of the \emph{n}-ratio, the
  consistency of the three tests is closer to zero when there is a
  \sout{negative} \color{blue}positive \color{black} correlation between
  the \emph{SD} and the \sout{mean} \color{blue}sample size
  \color{black} (meaning that the \color{blue}largest \color{black}
  group \sout{with the highest mean} has the \sout{lowest}
  \color{blue}largest \color{black} variance).''
\end{itemize}

\hypertarget{mise-en-forme-et-notations-1}{%
\subsubsection{Mise en forme et
Notations}\label{mise-en-forme-et-notations-1}}

Une légende est manquante pour certaines notations mathématiques. Par
exemple, en ce qui concerne l'équation (1), bien que \(n_j\), \(k\) et
\(s^2_j\) aient été correctement définis, les définitions pour
\(\bar{x_j}\), \(\bar{x_{..}}\) et \(N\) ne sont données que plus tard,
en référence Ã~ d'autres équations. Cela peut rendre la lecture de
l'article plus compliquée pour certaines personnes non familières avec
ces notations.

Par ailleurs, comme dans l'article précédent sur le test \(t\) de
Welch, on constate certaines incohérences en termes de notation. Par
exemple, si la moyenne de chaque groupe est définie par \(\bar{x_j}\)
dans l'équation (1), elle est définie par \(\bar{X_j}\) dans
l'équation (7).

On a également omis d'italiser certaines lettres statistiques (comme
dans les Figures par exemple).

\textbf{p.21} :
\sout{(\(SD_{spanish}=.80 > SD_{english}=.50\)\color{blue})}
(\(S_{spanish}=.80 > S_{english}=.50\)\color{blue}, with S = sample
standard deviation) {[}\ldots{]} \color{black}For men, the reverse was
true \sout{(\(SD_{spanish}=.97 < SD_{english}=1.33\))}
(\(S_{spanish}=.97 < S_{english}=1.33\))

Enfin, dû Ã~ un manque de connaissance de Latex lors de mes premières
tentatives d'écritures d'articles via Rmarkdown, certaines majuscules
sont manquantes dans les références bibliographiques.

\hypertarget{fautes-de-frappe-et-grammaire-et-autres}{%
\subsubsection{Faute(s) de frappe et grammaire et
autres}\label{fautes-de-frappe-et-grammaire-et-autres}}

\begin{itemize}
\tightlist
\item
  \textbf{p20} : ``which can lead\sout{s} to asymmetry in the
  distribution'';
\item
  \textbf{p.20} : ``\ldots{} we think that a \sout{first} realistic
  first step towards progress would be to get researchers\ldots{}'';
\item
  \textbf{p.21} : ``Based on mathematical explanations and Monte\sout{o}
  Carlo simulations'';
\item
  \textbf{p.22} : ``\sout{With} \color{blue}w\color{black}ith
  \(N=\)\ldots{}'';
\item
  \textbf{p.22} : ``\sout{Where} \color{blue}w\color{black}here
  \sout{\(x_j\)} \(\bar{x_j}\) and \(s^2_j\) are respectively the group
  mean and the group variance\ldots{}'';
\item
  \textbf{p.23} : ``the type I error rate of all
  test\color{blue}s\color{black}'';
\item
  \textbf{p.23} : ``When there are more than \sout{three}
  \color{blue}two \color{black}groups;
\item
  \textbf{p.24} : ``In Figure\color{blue}s \color{black} 2 to 6 (see
  Figure 1 for the legend)'';
\item
  p.24 : ``\ldots{} \sout{whatever} the correlation between the \(SD\)
  and the mean \color{blue}does not matter\color{black}'';
\end{itemize}

\hypertarget{annexe-c-erratum-de-larticle-why-hedges-g-based-on-the-non-pooled-standard-deviation-should-be-reported-with-welchs-t-test-chapitre-4}{%
\subsection{\texorpdfstring{Annexe C: erratum de l'article ``Why Hedge's
g* based on the non-pooled standard deviation should be reported with
Welch's \(t\)-test'' (Chapitre
4)}{Annexe C: erratum de l'article ``Why Hedge's g* based on the non-pooled standard deviation should be reported with Welch's t-test'' (Chapitre 4)}}\label{annexe-c-erratum-de-larticle-why-hedges-g-based-on-the-non-pooled-standard-deviation-should-be-reported-with-welchs-t-test-chapitre-4}}

\begin{itemize}
\tightlist
\item
  \textbf{p.39}: titre de la Table 1: ``\sout{Expectency}
  \color{blue}Expectation\color{black}, bias and variance of Cohen's
  \sout{\(d_s\)} \(d\)''
\item
  \textbf{p.39}: dans cette même table, les indices ``\(_s\)'' doivent
  être supprimés dans la première colonne.
\item
  \textbf{p.41}: les mêmes remarques (concernant le titre et les
  indices) s'appliquent Ã~ la table 2.
\item
  \textbf{p.46}: les mêmes remarques (concernant les indices)
  s'appliquent Ã~ la table 3. De plus, il convient d'écrire
  ``\sout{Cohen's} \color{blue}Hedges' \(g^*\)\color{black}''.
\end{itemize}
\end{appendix}
