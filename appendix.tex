\clearpage



\begin{appendix}
\hypertarget{annexes}{%
\section{Annexes}\label{annexes}}

\hypertarget{annexe-a-erratum-de-larticle-why-psychologists-should-by-default-use-welchs-t-test-instead-of-students-t-test-chapitre-2}{%
\subsection{\texorpdfstring{Annexe A: erratum de l'article ``Why
psychologists Should by Default Use Welch's \emph{t}-test Instead of
Student's \emph{t}-test'' (Chapitre
2)}{Annexe A: erratum de l'article ``Why psychologists Should by Default Use Welch's t-test Instead of Student's t-test'' (Chapitre 2)}}\label{annexe-a-erratum-de-larticle-why-psychologists-should-by-default-use-welchs-t-test-instead-of-students-t-test-chapitre-2}}

\hypertarget{erreurs-conceptuelles}{%
\subsubsection{Erreurs conceptuelles}\label{erreurs-conceptuelles}}

Nous spécifions Ã~ plusieurs reprises que le test \(t\) de Yuen
contrôle moins bien le taux d'erreur de type I que le test \(t\) de
Welch:\\
- p.14: \emph{``Yuen's \(t\)-test is not a good unconditional
alternative because we observe an unacceptable departure from the
nominal alpha risk of 5 percent for several shapes of distributions
{[}\ldots{]} particularly when we are studying asymmetric distributions
of unequal shapes''};\\
- p.15: \emph{``As it is explained in the additional file, Yuen's
\(t\)-test is not a better test than Welch's \(t\)-test, since it often
suffers high departure from the alpha risk of 5 percent''}.

Ceci n'est pas exact d'un point de vue purement statistique. A travers
le test de Yuen, on ne compare plus les moyennes de chaque groupe, mais
les moyennes \emph{trimmées} (soit les moyennes calculées sur les
données après avoir écarté les 20\% des scores les plus faibles
ainsi que les 20\% des scores les plus élevés). Or, Ã~ travers nos
simulations, les scénarios créés en vue de tester le taux d'erreur de
type \(I\) (risque alpha) étaient systématiquement des scénarios dans
lesquels les moyennes de chaque population étaient identiques. Lorsque
la distribution d'une population est parfaitement symétrique, la
moyenne et la moyenne trimmée seront identiques. Au contraire, lorsque
la distribution d'une population est asymétrique, la moyenne et la
moyenne trimmée diffèreront (la moyenne trimmée sera plus proche du
mode de la distribution et donc, représentera mieux cette dernière).

Notons malgré tout que d'un point de vue méthodologique, nous avons
déjÃ~ relevé que la plupart du temps, les chercheurs définissent
l'absence de différence entre les moyennes comme hypothèse nulle et
nos simulations démontrent que dans ce contexte, le test de Yuen n'est
pas approprié. En conclusion, le test de Yuen ne devrait être utilisé
que par des chercheurs ayant pleinement conscience du fait que les tests
\(t\) de Student et de Welch ne reposent pas sur la même hypothèse que
le test \(t\) de Yuen.

\hypertarget{commentaires-divers}{%
\subsubsection{Commentaires divers}\label{commentaires-divers}}

\begin{itemize}
\item
  p.9: nous décrivons 3 arguments en défaveur de l'usage du test de
  Levene. En troisième argument, nous mentionnons le manque de
  puissance du test de Levenne. Nous ne mentionnons cependant pas le
  fait qu'utiliser le test \(t\) de Student lorsque le test de Levene
  est non significatif revient Ã~ confondre le non rejet de l'hypothèse
  d'égalité des variances avec l'acceptation de l'hypothèse
  d'égalité des variances. Au sein du chapitre 5 sur les tests
  d'équivalence, il est démontré par simulation que même lorsqu'on
  s'assure d'avoir une puissance suffisante pour détecter une
  différence attendue, la stratégie qui consiste Ã~ interpréter le
  non rejet de l'hypothèse nulle comme un soutien en faveur de
  l'hypothèse nulle n'est pas appropriée.
\item
  p.12: nous mentionnons ceci : \emph{``When both variances and sample
  sizes are the same in each independent group, the \(t\)-values,
  degrees of freedom, and the \(p\)-values in Student's \(t\)-test and
  Welch's \(t\)-test are the same (see Table 1)\emph{. Avec du recul,
  cette phrase peut porter Ã~ confusion. Par "variances" il faut
  comprendre "}sample* variances'' ou ``variances \emph{estimates}''.
  Nous ne sommes donc }pas* en train de dire que les deux statistiques,
  ainsi que les degrés de liberté et \(p\)-valeurs qui leur sont
  associées seront identiques lorsque la condition d'homogénéité des
  variances sera respectée au niveau de la population, mais bien
  lorsque les estimations de chaque variance de population seront
  identiques.
\end{itemize}

\hypertarget{mise-en-forme-et-notations}{%
\subsubsection{Mise en forme et
Notations}\label{mise-en-forme-et-notations}}

Les lettres utilisées pour décrire les statistiques (test-\(t\) ou
test-\(F\)) doivent toujours être inscrites en \emph{italique}. Or,
cela a été omis Ã~ plusieurs reprises dans l'article. Par exemple, il
aurait fallu écrire:\\
- p.9: ``\ldots{} as the Mann-Whitney \(U\)-test\ldots{}'' au lieu de
``\ldots{} as the Mann-Whitney U-test\ldots{}'';\\
- p.9: ``\(F\)-ratio test'' au lieu de ``F-ratio test''.

Certaines notations mathématiques auraient également dû être
indiquées en italique. Par exemple, Ã~ la p.9, il aurait fallu
écrire:\\
- ``\(x_{ij}\)'' au lieu de ``\(\mathrm{x_{ij}}\)'';\\
- \textbar{}\(x_{ij}-\hat{\theta_j}\)\textbar{} au lieu de
\textbar{}\(\mathrm{x_{ij}-\hat{\theta_j}}\)\textbar.

Par ailleurs, il est très important d'être consistant dans le choix
des notations mathématiques, afin d'éviter d'embrouiller le lecteur.
Or, nous n'avons pas toujours respecté cela. Par exemple, nous avons
utilisé plusieurs notations différentes pour décrire l'écart-type et
la variance. Par exemple:\\
- p.9: nous utilisons respectivement SD1 et SD2 pour décrire
l'écart-type de chaque groupe;\\
- p.11 (équation 1): nous utilisons respectivement \(S^2_1\) et
\(S^2_2\) pour décrire la variance de chaque groupe;\\
- p.12 (équation 4): nous utilisons respectivement \(s^2_1\) et
\(s^2_2\) (lettres minuscules) pour décrire la variance de chaque
groupe.

C'est d'autant plus problématique qu'il y a parfois même des
inconsistances entre les notations utilisées dans les formules et
celles utilisées dans les légendes des formules. Par exemple, nous
spécifions p.11 que dans l'équation 1, \(s^2_1\) et \(s^2_2\) (lettres
minuscules) représentent les estimations de variance de chaque groupe
indépendant, alors qu'en réalité, les estimations des variances sont
représentées par \(S^2_1\) et \(S^2_2\) (lettres majuscules) dans
l'équation 1.

\hypertarget{fautes-de-frappe}{%
\subsubsection{Faute(s) de frappe}\label{fautes-de-frappe}}

\begin{itemize}
\tightlist
\item
  p.13: ``see \sout{v} \textbf{Figure} 2a''. \newpage \#\# Annexe B:
  erratum de l'article ``Taking parametric assumptions very seriously:
  Arguments for the Use of Welch’s \emph{F}-test instead of the
  Classical \emph{F}-test in One-Way ANOVA'' (Chapitre 3)
\end{itemize}

\hypertarget{mise-en-forme-et-notations-1}{%
\subsubsection{Mise en forme et
Notations}\label{mise-en-forme-et-notations-1}}

Une légende est manquante pour certaines notations mathématiques. Par
exemple, en ce qui concerne l'équation (1), bien que \(n_j\), \(k\) et
\(s^2_j\) aient été correctement définis, les définitions pour
\(\bar{x_j}\), \(\bar{x_{..}}\) et \(N\) ne sont données que plus tard,
en référence Ã~ d'autres équations. Cela peut rendre la lecture de
l'article plus compliquée pour certaines personnes non familières avec
ces notations.

Par ailleurs, comme dans l'article précédent sur le test \(t\) de
Welch, on constate certaines incohérences en termes de notation. Par
exemple, si la moyenne de chaque groupe est définie par \(\bar{x_j}\)
dans l'équation (1), elle est définie par \(\bar{X_j}\) dans
l'équation (7).

Enfin, dû Ã~ un manque de connaissance de Latex lors de mes premières
tentatives d'écritures d'articles via Rmarkdown, certaines majuscules
sont manquantes dans les références bibliographiques. S'assurer qu'une
lettre apparaisse en majuscule, via latex, implique de l'entourer des
symboles \{\}, ce qui n'a pas été fait. Par exemple, dans le titre de
l'article de Tiku(1971), il aurait fallu indiquer ``Power function of
the \{F\}-test..'' au lieu de ``Power function of the F-test\ldots{}''.
Cela ne serait pas arrivé, si j'avais utilisé un outil comme Zotero,
afin d'exporter directement un fichier au format Bibtex (puisque via ces
outils, ce genre de détail est automatiquement inclu), mais je n'ai
découvert cette possibilité que récemment.

\hypertarget{fautes-de-frappe-1}{%
\subsubsection{Faute(s) de frappe}\label{fautes-de-frappe-1}}

\begin{itemize}
\tightlist
\item
  p.18: ``Although it is important to make sure \sout{test}
  \textbf{that} assumptions are met'';\\
\item
  p.19: ``\ldots{} we think that a \sout{first} realistic first step
  towards progress would be to get researchers\ldots{}'';\\
\item
  p.20: ``Based on mathematical explanations and Monte\sout{o} Carlo
  simulations'';\\
\item
  p.21: ``\sout{With} \textbf{w}ith \(N=\)\ldots{}'';\\
\item
  p.21: ``\sout{Where} \textbf{w}here \sout{\(x_j\)} \(\bar{x_j}\) and
  \(s^2_j\) are respectively the group mean and the group
  variance\ldots{}'';\\
\item
  p.22: "\ldots{} negative pairings (the group with the \sout{smallest}
  \textbf{largest} sample size is extracted from the population with the
  smallest \(SD\));\\
\item
  p.22: ``the type I error rate of all test\textbf{s}'';\\
\item
  p.24: ``\ldots{} which is either more liberal or more conservative,
  depending on the \(SDs\) and \sout{\(SD\)} \textbf{sample sizes}
  pairing'';\\
  \newpage \#\# Annexe C: échanges avec Geoff Cumming, en vue
  d'améliorer l'article non publié ``Why Hedges’ \(g^*_s\) based on
  the non-pooled standard deviation should be reported with Welch’s
  \(t\)-test''
\end{itemize}

A COLLER ICI \newpage \#\# Annexe D: A VOIR SI ERRATUM DU CHP 5?
\end{appendix}
