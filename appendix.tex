\clearpage



\begin{appendix}
\hypertarget{annexes}{%
\section{Annexe(s)}\label{annexes}}

\hypertarget{annexe-a-erratum-des-articles}{%
\subsection{Annexe A: erratum des
articles}\label{annexe-a-erratum-des-articles}}

\hypertarget{why-psychologists-should-by-default-use-welchs-t-test-instead-of-students-t-test-chapitre-2}{%
\subsubsection{\texorpdfstring{Why psychologists Should by Default Use
Welch's \emph{t}-test Instead of Student's \emph{t}-test (Chapitre
2)}{Why psychologists Should by Default Use Welch's t-test Instead of Student's t-test (Chapitre 2)}}\label{why-psychologists-should-by-default-use-welchs-t-test-instead-of-students-t-test-chapitre-2}}

\hypertarget{erreurs-conceptuelles}{%
\paragraph{Erreurs conceptuelles}\label{erreurs-conceptuelles}}

\begin{itemize}
\tightlist
\item
  Erreur (p.~{[}99{]}): ``As it is explained in the additional file,
  Yuen's \(t\)-test is not a better test than Welch's \(t\)-test, since
  it often suffers high departure from the alpha risk of 5 percent''.
\end{itemize}

Explication: d'un point de vue purement statistique, Ã~ travers le test
de Yuen, on ne compare plus les moyennes de chaque groupe, mais les
moyennes \emph{trimmées} (soit les moyennes calculées sur les données
après avoir écarté les 20\% des scores les plus bas ainsi que les
20\% des scores les plus élevés). Or, Ã~ travers nos simulations, les
scénarios créés en vue de tester le taux d'erreur de type I (risque
alpha) étaient systématiquement des scénarios dans lesquels les
moyennes de chaque groupe étaient identiques. Lorsque les distributions
sont symétriques, les moyennes de populations sont identiques aux
moyennes trimmées des populations. Quand les distributions sont
asymétriques, par contre, la moyenne trimmée sera plus proche du mode
de la distribution que la moyenne. Cela explique pourquoi nous avions
noté ceci: "Yuen's \(t\)-test is not a good unconditional alternative
because we oserve an unacceptable departure from the nominal alpha risk
of 5 percent for several shapes of distributions {[}\ldots{]}
particularly when we are studying asymmetric distributions of unequal
shapes.

A partir du moment où cela ne correspond \emph{pas} Ã~ l'hypothèse
nulle du test de Yuen, il est erroné de conclure Ã~ une inflation du
risque alpha pour ce test sur base de nos scénarios. *D'un point de vue
méthodologique, par contre, nous pensons qu'il n'est pas clair pour
beaucoup de gens que l'hypothèse nulle du test de Yuen n'est pas la
même que celle du test t de Student, ce qui pourrait être source
d'ambiguïté. En conclusion, ce test ne devrait être utilisé que par
des personnes ayant pleinement conscience du fait que le test de Student
et celui de Yuen ne testent pas la même hypothèse. **Relire l'article
de Wilcox et celui de Erceg-Hurn pour voir si c'est vraiment bien clair
dedans, histoire de me défendre un peu*

When both the normality and equal variances assumptions are violated, we
can use a combination of the Trimmed Means t-Test and Welch’s t-Test,
called the Yuen-Welch Test. Using the notation for the Trimmed Means
t-test, the Yuen-Welch Test is

\hypertarget{mise-en-forme-et-notations}{%
\paragraph{Mise en forme et
Notations}\label{mise-en-forme-et-notations}}

En termes de mises en forme, nous avons omis Ã~ plusieurs endroit
d'italiser les notations mathématiques. Par exemple, Ã~ la page
{[}93{]}, nous avons indiqué ``F-ratio test'' au lieu de ``\(F\)-ratio
test'' Ã~ plusieurs reprises. A la même page, nous avons également
noté ``\(\mathrm{x_{ij}}\)'' au lieu de ``\(x_{ij}\)'', et
\textbar{}\(\mathrm{x_{ij}-\hat{\theta_j}}\)\textbar{} au lieu de
\textbar{}\(x_{ij}-\hat{\theta_j}\)\textbar.

En termes de notation, nous avons relevé des inconsistantes dans la
notation de la variance de chaque groupes. Par exemple, dans l'équation
1, nous utilisons \(S^2_j\), alors que nous utilisons \(s^2_j\) dans
l'équation 4 ou encore SDj lorsque nous définission la statistique du
\(F\)-ratio Ã~ la page {[}93{]}.

\hypertarget{taking-parametric-assumptions-very-seriously-arguments-for-the-use-of-welchuxe2s-f-test-instead-of-the-classical-f-test-in-one-way-anova-chapitre-3}{%
\subsubsection{\texorpdfstring{Taking parametric assumptions very
seriously: Arguments for the Use of Welch’s \emph{F}-test instead of
the Classical \emph{F}-test in One-Way ANOVA (Chapitre
3)}{Taking parametric assumptions very seriously: Arguments for the Use of Welch’s F-test instead of the Classical F-test in One-Way ANOVA (Chapitre 3)}}\label{taking-parametric-assumptions-very-seriously-arguments-for-the-use-of-welchuxe2s-f-test-instead-of-the-classical-f-test-in-one-way-anova-chapitre-3}}

\hypertarget{erreurs-conceptuelles-1}{%
\paragraph{Erreurs conceptuelles}\label{erreurs-conceptuelles-1}}

\hypertarget{mise-en-forme-et-notations-1}{%
\paragraph{Mise en forme et
Notations}\label{mise-en-forme-et-notations-1}}

Dû Ã~ un manque de connaissance de Latex lors de mes premières
tentatives d'écritures d'articles via Rmarkdown, certaines majuscules
sont manquantes dans les références bibliographiques. S'assurer qu'une
lettre apparaissent en majuscule, via latex, implique de l'entourer des
symboles \{\}, ce qui n'a pas été fait. Par exemple, dans le titre de
l'article de Tiku(1971), il aurait fallu indiquer ``Power function of
the \{F\}-test..'' au lieu de ``Power function of the F-test\ldots{}''.
Cela ne serait pas arrivé, si j'avais utilisé un outil comme Zotero,
afin d'exporter directement un fichier au format Bibtex (puisque via ces
outils, ce genre de détail est automatiquement inclu), mais je n'ai
découvert cette possibilité que très récemment.

\hypertarget{effect-sizes}{%
\subsubsection{Effect sizes}\label{effect-sizes}}

\hypertarget{equivalence-tests}{%
\subsubsection{Equivalence tests}\label{equivalence-tests}}

\newpage

\hypertarget{annexe-b}{%
\subsection{Annexe B}\label{annexe-b}}

Insert code (if any) used during your dissertation work here.
\end{appendix}
